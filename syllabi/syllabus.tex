\documentclass[12pt]{article}
\usepackage[margin=0.75in]{geometry}
\usepackage{graphicx}
\usepackage{longtable}
\setlength{\parindent}{0mm}

\begin{document}

{\centering
\large University of North Georgia \par
\large College of Science and Mathematics \par
\large Department of Physics \par
\large PHYS 1111 - Introductory Physics I - Summer 2018 \par
}
\hfill \break \vspace{-4mm}

``The noblest pleasure is the joy of understanding.'' -Leonardo da Vinci
\hfill \break

\underline{\textbf{General Information}} \par
Instructor: Dr. Nathan Harrison \par
Office: Science 113 \par
Email: Standard UNG email address \par
Office Hours: See the pdf file of my schedule.
\hfill \break

\underline{\textbf{Required Materials}} \par
Textbook: College Physics 11th Edition w/ Webassign by Serway \par
Scientific or graphing calculator, ruler, protractor, paper, scan-tron forms \par
GitHub account \par
SageMath (Cloud account or your own installation) \par
Java JDK 1.8 or greater
\hfill \break

\underline{\textbf{Course Description}} \par
This is an introductory course which will include material from mechanics, thermodynamics, and waves;
and is the first course in a two-semester Physics sequence.
Elementary algebra and trigonometry will be used.
The prerequisite is MATH 1113 or MATH 1450.
The lecture is three credits while the lab is one credit. (see the lab syllabus for more details)
\hfill \break

\underline{\textbf{Course Content}} \par
We will be covering Parts 1-3 (Mechanics, Thermodynamics, and Vibrations and Waves) of the textbook which consists of 14 chapters.
We will cover approximately 2-3 chapters per week and one corresponding lab per chapter (see the lab syllabus for more details). A detailed schedule can be found at the end of this syllabus.
\hfill \break

\underline{\textbf{Expected Course Outcomes}} \par
The objective of the course is to have the student learn and be proficient in the application of the basic laws of mechanics, thermodynamics, and waves.
After having taken this course, the student should be able to:
\begin{enumerate}
\item interpret physical situations as stated in word problems
\item be able to identify the physical laws appropriate to the physical situation at hand
\item be able to use mathematics/physical law as a tool for predicting behavior of systems
\item be able to use computers/sensors as a tool for experimental investigation of physical law
\item be able to represent physical systems in multiple representations, e.g. mathematically, pictorially, graphically, etc.
\item be able to use various technologies as tools for scientific endeavor
\end{enumerate}
\hfill \break

\underline{\textbf{Means of Assessment and Grading Scheme}} \par
There will be several exams during the semester as well as several short quizzes.
Exams will be announced at least one week in advance, quizzes may occasionally be unannounced.
You will be provided with a ``clicker'' for answering in-class polls and practice problems; clicker questions will count as part of your participation grade.
Homework assignments will be given on a fairly regular basis via WebAssign.
The final exam is comprehensive and mandatory. \par
\hfill \break
Average of quizzes and exams: 50\% \par
WebAssign HW: 25\% \par
Participation: 25\%
\hfill \break

\underline{\textbf{Supplemental Syllabus}} \par
Please see https://ung.edu/academic-affairs/policies-and-guidelines/supplemental-syllabus.php for the following information:
\begin{enumerate}
\item Academic Exchange
\item Academic Integrity Policy
\item Academic Success Plan Program
\item Class Evaluations
\item Course Grades and Withdrawal Process
\item Disruptive Behavior Policy
\item Inclement Weather
\item Smoking Policy
\item Students with Disabilities
\end{enumerate}

\underline{\textbf{Additional Information}} \par
\begin{enumerate}
\item You will have to self-register with WebAssign in order to access the online homework.
A “class key” is need to do this, it will be emailed to you during the first week of classes.
%\item Several students may take this course for honors credit.
%Since space is limited, speak up soon if interested.
\item This syllabus may be adjusted if deemed necessary by the instructor.
\end{enumerate}

\pagebreak

\underline{\textbf{Schedule (approximate)}} \par
* See the university schedule for important dates such as drop deadlines, etc.
\begin{longtable}{| p{0.07\textwidth} | p{0.07\textwidth} | p{0.4\textwidth} |}
\hline
Week & Date & Topics/Reading \\ \hline
1 & 5/28 & ch. 1-3 \\ \hline
2 & 6/4 & ch. 4-6 \\ \hline
3 & 6/11 & Review, exam, ch. 7-8 \\ \hline
4 & 6/18 & ch. 9-11 \\ \hline
5 & 6/25 & ch. 12-14, lecture \& lab finals \\ \hline
\end{longtable}

\end{document}
